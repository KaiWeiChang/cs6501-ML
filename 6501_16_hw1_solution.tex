\documentclass[12pt,fullpage,letterpaper]{article}

\newenvironment{proof}{\noindent{\bf Proof:}}{\qed\bigskip}

\newtheorem{theorem}{Theorem}
\newtheorem{corollary}{Corollary}
\newtheorem{lemma}{Lemma} 
\newtheorem{claim}{Claim}
\newtheorem{fact}{Fact}
\newtheorem{definition}{Definition}
\newtheorem{assumption}{Assumption}
\newtheorem{observation}{Observation}
\newtheorem{example}{Example}
\newcommand{\qed}{\rule{7pt}{7pt}}


\newcommand{\assignment}[4]{
\thispagestyle{plain} 
\newpage
\setcounter{page}{1}
\noindent
\begin{center}
\framebox{ \vbox{ \hbox to 6.28in
{\bf CS6501: Advanced Machine Learning \hfill #1}
\vspace{4mm}
\hbox to 6.28in
{\hspace{2.5in}\large\mbox{Problem Set #2}}
\vspace{4mm}
\hbox to 6.28in
{{\it Handed Out: #3 \hfill Due: #4}}
}}
\end{center}
}

\newcommand{\handout}[3]{
\thispagestyle{plain} 
\newpage
\setcounter{page}{1}
\noindent
\begin{center}
\framebox{ \vbox{ \hbox to 6.28in
{\bf CS6501-01 Advanced Machine Learning  \hfill #1}
\vspace{4mm}
\hbox to 6.28in
{\hspace{2.5in}\large\mbox{#2}}
\vspace{4mm}
\hbox to 6.28in
{{\it Date: #3 \hfill Name (NetID):\makebox[2in]{\hrulefill}}}
}}
\end{center}
}

\newcommand{\solution}[4]{
\thispagestyle{plain} 
\newpage
\setcounter{page}{1}
\noindent
\begin{center}
\framebox{ \vbox{ \hbox to 6.28in
{\bf CS6501 - Advanced Machine Learning \hfill #4}
\vspace{4mm}
\hbox to 6.28in
{\hspace{2.5in}\large\mbox{Problem Set #3}}
\vspace{4mm}
\hbox to 6.28in
{#1 \hfill {\it Handed In: #2}}
}}
\end{center}
\markright{#1}
}

\newenvironment{algorithm}
{\begin{center}
\begin{tabular}{|l|}
\hline
\begin{minipage}{1in}
\begin{tabbing}
\quad\=\qquad\=\qquad\=\qquad\=\qquad\=\qquad\=\qquad\=\kill}
{\end{tabbing}
\end{minipage} \\
\hline
\end{tabular}
\end{center}}

\def\Comment#1{\textsf{\textsl{$\langle\!\langle$#1\/$\rangle\!\rangle$}}}


\usepackage{graphicx}
\usepackage{subfigure}
\usepackage{amsmath}
\usepackage{amssymb}
%\usepackage{qtree}
\usepackage{epsfig}
\usepackage{enumerate}
\usepackage{color}
\usepackage{algorithmic}
\usepackage{hyperref}
\usepackage{color}

%\usepackage{parskip}

\sloppy
\parskip = 0.5cm

\newcommand{\ignore}[1]{}
\newcommand{\pp}{\noindent}
\newcommand{\ov}{\overline}
\newcommand{\bb}[1]{{\bf #1}}
\renewcommand{\labelitemii}{\tiny$\circ$}

\newcommand{\question}[1]{#1}%{}
%\newcommand{\answer}[2]{#1}
%\input{answerdef.tex}
%\newcommand{\answer}[2]{#1}
\newcommand{\answer}[2]{{
\vspace{10pt} 
\color{red}{#2}
\vspace{10pt}
}
}
\newcommand{\comment}[1]{}


\oddsidemargin 0in
\evensidemargin 0in
\textwidth 6.5in
\topmargin -0.5in
\textheight 9.0in

\begin{document}
\setlength{\unitlength}{1mm}

\thispagestyle{plain}
\newpage
\assignment{2017}{1}{Feb 11, 2017}{Feb 18, 2017}

\begin{itemize}
\item Feel free to talk to other members of the class in doing the homework. You should, however,
write down your solution yourself.  Please try to keep the solution brief and clear.

\item Please use Piazza first if you have questions about the homework. Also feel free to come to office hours.

\item Please, no handwritten solutions. You will submit your solution manuscript as a single pdf file. Please use \LaTeX to typeset your solutions.

\item The homework is due at 11:59 PM on the due date. We will be using
Collab for collecting the homework assignments. Please submit your solution manuscript as a pdf file and your code as a zip file.  Please do NOT hand in a hard copy of your write-up.
Contact the TAs if you are having technical difficulties in 
submitting the assignment. 
\end{itemize}



\section{Binary Classification} 
A {\em linear program} can be stated as follows:
\begin{definition}
Let 
$A$ be an $m \times n$ real-valued matrix,
$\vec{b} \in \mathbb{R}^m$, and $\vec{c} \in \mathbb{R}^n$.
Find a $\vec{t} \in \mathbb{R}^n$, that minimizes the linear function
\begin{eqnarray*}
  & & z(\vec{t}) = \vec{c}~^T \vec{t} \\
\textrm{subject to} & & A \vec{t} \geq \vec{b}
\end{eqnarray*}
\end{definition}

In the linear programming terminology, $\vec{c}$ is often referred
to as the {\em cost vector} and $z(\vec{t})$ is referred to as the {\em objective
function}.\footnote{Note that you don't need
to really know how to solve a linear program since you can use Matlab or Scipy to obtain the
solution (although you may wish to brush up on Linear Algebra). }
We can use this framework to define the problem of learning a linear
discriminant function.\footnote{This discussion closely
  parallels the linear programming representation found in 
  {\em Pattern Classification}, by Duda, Hart, and Stork.}

\textbf{The Learning Problem:}\footnote{Note that the notation used in the
Learning Problem is
{\bf unrelated} to the one used in the Linear Program definition. You may want to
figure out the correspondence.} \hspace{2mm}
Let $\vec{x_1}, \vec{x_2}, \ldots, \vec{x_m}$ 
represent $m$ samples, where each sample $\vec{x_i}\in \mathbb{R}^n$ is an $n$-dimensional
vector, and $\vec{y} \in \{-1, 1\}^m$ is an $m \times 1$
vector representing the respective labels of each of the $m$ samples. Let
$\vec{w} \in \mathbb{R}^n$ be an $n \times 1$ vector representing the weights of the
linear discriminant function, and $\theta$ be the threshold value. 

We {\em predict} $\vec{x_i}$ to be a {\em positive} example if
$\vec{w}^T \vec{x_i} + \theta \geq 0$. On the other hand, we {\em predict}
$\vec{x_i}$ to be a {\em negative} example if $\vec{w}^T \vec{x_i} + \theta < 0$.

We hope that the learned linear function can separate the data set.  
That is, for the true labels we hope
\begin{equation}
\label{eq:separable}
y_i = \begin{cases}
 1 & \mbox{if } \vec{w}^T \vec{x_i} + \theta \ge 0 \\
-1 & \mbox{if } \vec{w}^T \vec{x_i} + \theta < 0. \\
\end{cases}
\end{equation}
%\begin{equation}
%\textrm{if } \left\{
%\begin{array}{cc}
%\vec{w}^T \vec{x_i} + \theta \geq 0, & y_i = 1 \\
%\vec{w}^T \vec{x_i} + \theta < 0, & y_i = -1
%\end{array}
%\right. \label{eq:separable}
%\end{equation}

In order to find a good linear separator, we propose the following linear program:
\begin{eqnarray}
  \min_{\vec{w}, \theta, \delta} & & \delta  \label{eq:lin_prog_discriminant_obj}\\
 \textrm{subject to } & & y_i(\vec{w}^T \vec{x_i} + \theta) \geq 1 - \delta, \qquad \forall (\vec{x_i},y_i) \in D  \label{eq:lin_prog_discriminant_constraint}\\
  & & \delta \geq 0  \label{eq:lin_prog_discriminant_bound}
\end{eqnarray}
where $D$ is the data set of all training examples.
\begin{enumerate}

  
  \item [a.] {\em[10 points]} A data set
      $D=\{(\vec{x_i},y_i)\}_{i=1}^m$ that satisfies 
      condition (\ref{eq:separable}) above is called
      {\em linearly separable}. 
      {\bf Show that the data set
      $D$ is linearly separable 
      if and only if there exists
      a hyperplane ($\vec{w'}, \theta'$) that satisfies 
      condition
      \eqref{eq:lin_prog_discriminant_constraint} with $\delta = 0$} 
	  (Need a hint?\footnote{{\bf Hint:} Assume that $D$ is linearly separable. $D$ is a
      finite set, so it has a positive example that is {\em closest}
      to the hyperplane among all positive examples. Similarly, there
      is a negative example that is {\em closest} to the hyperplane
      among negative examples. Consider their distances and use them
      to show that condition
      (\ref{eq:lin_prog_discriminant_constraint}) holds. Then show the
      other direction.
	  }).
      Conclude that $D$ is linearly separable iff the optimal solution
      to the linear program
      [\eqref{eq:lin_prog_discriminant_obj} to \eqref{eq:lin_prog_discriminant_bound}]
      attains $\delta = 0$.
      {\bf What can we say about the linear separability of the data set
      if there exists a hyperplane that satisfies condition
      \eqref{eq:lin_prog_discriminant_constraint} with $\delta > 0$?}


{
\color{red}
  The solution to (a.) contains three parts. \\
     
      \noindent \textbf{(i) Linear Separability $\Rightarrow \delta = 0$:}


The separability property implies that there exists a hyperplane $\vec{v}^T \vec{x} + \rho$ such that
    \begin{equation*}
      \min_{\substack{(\vec{x},y)\in D \\ y=1}} (\vec{v}^T \vec{x} + \rho) \quad \geq \quad 0 \quad > \quad \max_{\substack{(\vec{x},y)\in D \\ y=-1}} (\vec{v}^T \vec{x} +\rho)
    \end{equation*}
    Let $\vec{x_i}$ to be the positive sample that is closest to the hyperplane
    $\vec{v}^T\vec{x} + \rho$, and 
    \begin{equation*}
      p^{+} = \min_{\substack{(\vec{x},y)\in D \\ y=1}} (\vec{v}^T \vec{x} +
      \rho) = (\vec{v}^T \vec{x_i} + \rho)
  \end{equation*}
Similarly, let $\vec{x_j}$ to be
    the negative sample that is closest to the hyperplane $\vec{v}^T\vec{x} + \rho$, and
    \begin{equation*}
    p^{-} = \max_{\substack{(\vec{x},y)\in D \\ y=-1}} (\vec{v}^T \vec{x} + \rho) = (\vec{v}^T
    \vec{x_j} + \rho)
  \end{equation*}
    Note that from the definition of separability, we know that $p^{+}$ and $p^{-}$ exist
    and $p^{+} \geq 0 > p^{-}$. 

    {\bf Step 1: Shifting the hyperplane.}
    Since $p^{+} \geq 0 > p^{-}$, there exist $\eta \geq 0$ such that
    $p^{+} - \eta \geq 0 > p^{-} - \eta$. Hence, for some values of $\eta$,
    the hyperplane $\vec{v}^{T} \vec{x} + \rho - \eta = 0$ also separates $D$.
    We will find the value of $\eta$ for which the hyperplane  
    $\vec{v}^{T} \vec{x} + \rho - \eta = 0$ is equi-distant from $\vec x_{i}$ and $\vec x_{j}$
    and separates $D$.

    The value of $\eta$ can be obtained as follows:
    \begin{equation}
      \begin{split}
        \frac{|\vec{v}^{T} \vec{x_i} + \rho - \eta|}{\|\vec{v}\|} &= 
        \frac{|\vec{v}^{T} \vec{x_j} + \rho - \eta|}{\|\vec{v}\|}, \\
        \vec{v}^T \vec{x_i} + \rho - \eta &= -(\vec{v}^T \vec{x_j} + \rho - \eta),\\
        p^{+} - \eta &= - p^{-} + \eta.
      \end{split}
    \end{equation}
    This implies that $\eta = \frac{p^{+} + p^{-}}{2}$.
    We can easily verify that for such $\eta$,
    $p^{+}-\eta \geq 0 > p^{-} - \eta$,
    and hence the resulting hyperplane $\vec{v}^{T} \vec{x} + \rho - \eta = 0$
    separates $D$.

      { \bf Step 2: Normalizing the hyperplane.} 
      With the new hyperplane \\ 
      $\vec{v}^T
      \vec{x} + \rho - \eta = 0$,
      \begin{eqnarray*}
        \min_{\substack{(\vec{x},y)\in D \\ y=1}} (\vec{v}^T \vec{x} + \rho - \eta) & = & \frac{p^{+}-p^{-}}{2} \\
      \textrm{and } \qquad 
        \max_{\substack{(\vec{x},y)\in D \\ y=-1}} (\vec{v}^T \vec{x} + \rho - \eta) & = & \frac{p^{-}-p^{+}}{2} \\
        \therefore y(\vec{v}^T \vec{x} + \rho - \eta) & \geq & \frac{p^{+}-p^{-}}{2} \qquad \forall (\vec{x},y) \in D
      \end{eqnarray*}

      Given that $p^{+}>p^{-}$ and $\eta = \frac{p^{+}+p^{-}}{2}$, by setting $\vec{w} =
      \frac{\vec{v}}{\eta}$, $\theta = \frac{\rho - \eta}{\eta}$, and $\delta=0$,
      \begin{equation*}
        y(\vec{w}^T \vec{x} + \theta) \geq 1 - \delta, \quad \forall (\vec{x},y) \in D
      \end{equation*}
      \noindent \textbf{(ii) Linear Separability $\Leftarrow \delta = 0$:}

      If there exists a hyperplane $\vec{w}^T \vec{x} + \theta$ such that
      \begin{equation*}
        y(\vec{w}^T \vec{x} + \theta) \geq 1, \quad \forall (\vec{x},y) \in D
      \end{equation*}
      It is trivial to show that
      \begin{eqnarray*}
        (\vec{w}^T \vec{x} + \theta) \geq 1 & \geq &  0, \qquad \forall (\vec{x},y)\in D, y=1 \\
      \textrm{and } \qquad (\vec{w}^T \vec{x} + \theta) \leq -1 & < &  0, \qquad \forall (\vec{x},y)\in D, y=-1
      \end{eqnarray*}

	\noindent \textbf{(iii) When $\delta>0$:}

      Now consider the value of $\delta$. Note that if $1-\delta > 0$, we can apply similar argument and show that the data set is linear separable. If $\delta \geq 1$, we are not sure if the data set is separable or not. If the {\em minimal} $\delta \geq 1$, then the data set is not separable.


}

  \item [b.] {\em [5 points]} {\bf Show that there is a
      trivial optimal solution for the following linear
      program:}
    \begin{eqnarray*}
      \min_{\vec{w},\theta,\delta} & & \delta  \\
%\min_{\delta} & & \delta  \\
      \textrm{subject to } & & y_i(\vec{w}^T \vec{x_i} + \theta) \geq - \delta, \qquad \forall (x_i,y_i) \in D \\
      && \delta \geq 0  \\
    \end{eqnarray*}
	{\bf Show the optimal solution and use it to (very briefly)
	explain why we chose to formulate the linear program as
	[\eqref{eq:lin_prog_discriminant_obj} to \eqref{eq:lin_prog_discriminant_bound}]
	instead.}
	
	{\color{red}
	
The trivial solution is
      \begin{equation*}
        \vec{w} = \vec{0}, \quad \theta=0, \quad \delta = 0.
      \end{equation*}
      This solution is optimal because 1) the constraints are
      satisfied, and 2) zero is the best value we can get for
      $\delta$.  We do not use this formulation because the solver
      might give us this optimal solution which does not give us a
      hyperplane at all. Note that there exists other optimal solutions for
      this formulation.
	
	}

\item [c.] {\em [10 points]} 
Let $\vec{x_1} \in \mathbb{R}^n$, $\vec{x_1}^T =
\begin{bmatrix}
  1 & 1 & \ldots & 1
\end{bmatrix}$ and $y_1 = 1$. 
Let
$\vec{x_2} \in \mathbb{R}^n$, $\vec{x_2}^T =
\begin{bmatrix}
  -1 & -1 & \ldots & -1
\end{bmatrix}$ and $y_2 = -1$.
The data set $D$ is defined as
\begin{equation*}
    D = \{ (\vec{x_1},y_1), (\vec{x_2},y_2)\}.
\end{equation*}
Consider the formulation in
[\eqref{eq:lin_prog_discriminant_obj} to \eqref{eq:lin_prog_discriminant_bound}]
applied to
$D$. {\bf Show the set of all possible optimal solutions (solve this problem by hand)}.
  
  \end{enumerate}

{
\color{red}
Given that there are only two samples in the data set, the data
      set is separable. Therefore, the optimal $\delta$ is 0. 
      Now our job becomes finding $\vec{w}$ and $\theta$ that satisfy the following constraints:
      \begin{eqnarray*}
        w_1 + w_2  \ldots +w_n  + \theta  & \geq & 1 \\
        -(-w_1 - w_2  \ldots -w_n  + \theta) & \geq & 1 \\
	& & \\
      \therefore \qquad 
        w_1 + w_2 + \ldots + w_n  & \geq & 1 + | \theta|
      \end{eqnarray*}
      
      Hence, $(\vec{w},\theta,\delta)$ is the optimal solution
      if $\delta = 0$ and $w_1 + w_2 + \ldots + w_n  \geq  1+ | \theta|$.    
}

\section{Multiclass Classification} 
In this problem set, we will derive, implement, and test an SGD method for a multi-class SVM classification model, and then we will compare it with the one-vs-all approach. 

Given a data set $D = \{x_i, y_i\}_{i=1}^N$ with $K$ classes, the multi-class SVM can be written as the following  unconstrained optimization problem:
\begin{equation}
\label{eq:multiclass2}
    \min_{w=\{w_k\}} \quad \frac{1}{2} \sum_{k=1}^K w_k^T w_k + C \sum_{i=1}^N \left( \max_{k=1\ldots K} \left(\Delta(y_i, k)+w_k^Tx_i\right)-w^T_{y_i} x_i\right),
\end{equation}
where  $w_k$ is the weight vector for class $k$, ($x_i$, $y_i$) is the $i^{th}$ instance and 
\begin{equation*}
\Delta(y,k) = 
    \begin{cases}
    1 & \mbox{if } y\neq k,\\
    0 & \mbox{otherwise}.
    \end{cases}
\end{equation*}

To derive the SGD  algorithm, we first need to rewrite the objective function in Eq. \eqref{eq:multiclass2} into the form of $\sum_i g_i(w)$:
\begin{equation*}
\label{eq:multiclass}
    \min_{w=\{w_k\}} \quad \sum^N_{i=1} \left(\frac{1}{2N} \sum^K_{k=1} w_k^T w_k + C  \left( \max_{k=1\ldots K} \left(\Delta(y_i, k)+w_k^Tx_i\right)-w^T_{y_i} x_i\right)\right),
\end{equation*}
so that 
\begin{equation*}
    g_i(w) =  \left(\frac{1}{2N} \sum^K_{k=1} w_k^T w_k + C  \left( \max_{k=1\ldots K} \left(\Delta(y_i, k)+w_k^Tx_i\right)-w^T_{y_i} x_i\right)\right). 
\end{equation*}

Now let's derive the sub-gradient of $g_i(w)$.
Consider $\partial g_i(w)/ \partial w_k$, the partial sub-gradient of the first term of $g_i(w)$ is $w_k/N$.
Regarding the second term, we have to consider several cases. 
Let $$\tilde{y} = \arg\max_k \left(\Delta(y_i, k)+w^T_k x_i\right)$$. If $\tilde{y} = k$ and $y_i = k$, the partial sub-gradient of the second term is $C(x - x) = 0$. Therefore, 
$\partial g_i(w)/ \partial w_k = w_k/N$ when $\tilde{y} = k$ and $y_i = k$. 

How about other cases?

\begin{enumerate}
\item [a.] {[15 points]} Please complete the following formulation
\begin{equation}
\label{eq:grad}
\frac{\partial g_i(w)}{\partial w_k} = \begin{cases}
w_k/N     \hspace{5cm} & \mbox{if } \tilde{y} = k \mbox{ and } y_i = k\\
\\
{\color{red} w_k/N -Cx_i}    &\mbox{if } \tilde{y} \neq k \mbox{ and } y_i = k\\
       \\
     {\color{red} w_k/N + Cx_i}    &\mbox{if } \tilde{y} = k \mbox{ and } y_i \neq k\\
       \\
        {\color{red} w_k/N}    &\mbox{if } \tilde{y} \neq k \mbox{ and } y_i \neq k\\
\end{cases}
\end{equation}




\item [b.]  {[15 points]} The SGD algorithm runs as follows:  

\begin{algorithmic}
\STATE \FOR {epoch = 0 $\ldots$ T}
\FOR {$(\vec{x_i}, y_i)$ in $D$}
 \FOR {k= 1 $\ldots$ K}
   \STATE {$w_k\leftarrow w_k - \eta \  \partial g_i(w_k) / \partial w_k$}
\ENDFOR
\ENDFOR
\ENDFOR
\end{algorithmic}
$\eta$ is the step size. The step size is a user specified parameter. Here, we can set it with a default value 0.01.


Plug Eq. \eqref{eq:grad} in, we have:
\begin{algorithmic}
\STATE \FOR {epoch = 0 $\ldots$ T}
\FOR {$(\vec{x_i}, y_i)$ in $D$}
\STATE { $\tilde{y} = \arg\max_k \left(\Delta(y_i, k)+w^T_k x_i\right)$}
\FOR {k= 1 $\ldots$ K}
\STATE
   \STATE {$w_k\leftarrow w_k - \eta ({\color{red} w_k/N } ) $}
   
\ENDFOR
 \IF {$\tilde{y} \neq {y_i}$}
 \STATE
   \STATE {$w_{y_i}\leftarrow w_{y_i} - \eta ({\color{red}  -Cx_i} )$}
    \STATE
    \STATE {$w_{\tilde{y}}\leftarrow w_{\tilde{y}} - \eta ({\color{red}  +Cx_i} )  $}

\ENDIF
\ENDFOR
\ENDFOR
\end{algorithmic}
Please, complete the above algorithm.

\item[c.] {[20points]} Let's implement a baseline one-vs-all  multiclass algorithm and conduct experiments on MNIST dataset~\footnote{http://yann.lecun.com/exdb/mnist/}.  We provide a basic code framework at \url{https://github.com/uvanlp/HW1} to facilitate the implementation. However, you're welcomed to write your own code. If you want to use the code we provided, implement the one-vs-all algorithm in the main function of one\_vs\_all\_with\_data\_reader.py. Then, conduct cross-validation to find the best C for all underlying binary classifier (assume the same C is used for all the binary classifiers). Note: You can use the LinearSVC class at Scikit-Learn,  \url{http://scikit-learn.org/stable/modules/generated/sklearn.svm.LinearSVC.html}, to implement the underlying classifiers.

Report the best C and the final accuracy.

\item[d.] {[25points]} Implement the SGD algorithm for multi-class SVM. We provide `multiclass\_svm.py' as the code framework. However, you have to implement the update rules. Conduct cross-validation to find the best C and report the result on the test set.\footnote{ To test if your algorithm implementation is correct, you can check if the objective function value of the SGD algorithm matches with the score from liblinear  https://www.csie.ntu.edu.tw/~cjlin/liblinear. Use the following command line to train a multi-class SVM model
``./train  -s 4 -B -1 -e 0.01 -C 1 train.data'' with C=1.
There is a function in the sample code allows you to print data in liblinear format. Check if the objective function values of your implementation is the same as that of liblinear on a small subset  (e.g., run on only the first 100 samples).}
\item[e.] {[0-10 points]} We will give bonus points for additional  experiments.
\end{enumerate}

\end{document}
